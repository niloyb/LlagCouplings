\documentclass{article}
\usepackage{lmodern}
\usepackage[margin=1in]{geometry}
\usepackage[utf8]{inputenc} % allow utf-8 input
\usepackage[T1]{fontenc}    % use 8-bit T1 fonts
%\usepackage{hyperref}       % hyperlinks
\usepackage{url}            % simple URL typesetting
\usepackage{booktabs}       % professional-quality tables
\usepackage{amsfonts}       % blackboard math symbols
\usepackage{nicefrac}       % compact symbols for 1/2, etc.
\usepackage{microtype}      % microtypography
\usepackage{natbib}

\usepackage{amsmath}
\usepackage{amsthm}
\usepackage{color}
\usepackage{graphicx}
\usepackage{setspace}
\usepackage{esint}
\usepackage{listings}
\lstset{language=R,
    basicstyle=\small\ttfamily,
    stringstyle=\color{DarkGreen},
    otherkeywords={0,1,2,3,4,5,6,7,8,9},
    morekeywords={TRUE,FALSE},
    deletekeywords={data,frame,length,as,character},
    keywordstyle=\color{black},
    commentstyle=\color{DarkGreen},
}

\usepackage[unicode=true,bookmarks=true,bookmarksnumbered=false,bookmarksopen=false,breaklinks=false,pdfborder={0 0 0},pdfborderstyle={},backref=page,colorlinks=true]{hyperref}
\hypersetup{pdftitle={Estimating Convergence},pdfauthor={Biswas Jacob},linkcolor=RoyalBlue,citecolor=RoyalBlue}
\usepackage[dvipsnames,svgnames,x11names,hyperref]{xcolor}


% Added Thm, Lemmas, etc.
\newtheorem{Th}{Theorem}[section]
\newtheorem{Lemma}[Th]{Lemma}
\newtheorem{Cor}[Th]{Corollary}
\newtheorem{Prop}[Th]{Proposition}
\newtheorem{Asm}[Th]{Assumption}
\newtheorem{Remark}[Th]{Remark}
\newtheorem{Def}[Th]{Definition}

\usepackage[ruled]{algorithm2e}
\usepackage{graphicx}
\usepackage{graphics}


\title{Erratum for Figure 5 of \citeauthor{biswas_2019}}

\author{Niloy Biswas \& Pierre E. Jacob}

\begin{document}
\maketitle

There was a bug in the code used to produce Figure 5 of \citet{biswas_2019}. 
The choice of step-size for the MALA and ULA chains was not set in the code as it was described in 
the paper. The corresponding script 
\href{https://github.com/niloyb/LlagCouplings/blob/master/ula_versus_mala/ula_versus_mala.R}{ula\_versus\_mala/ula\_versus\_mala.R} 
has now been corrected.  Below we provide details and an updated  Figure 5.  

\section{Choice of step-size in previous code}
\begin{itemize}
  \item Langevin Monte Carlo methods are based on a proposal of the form 
    \begin{equation} \label{eq:prop1}
        Y = X + \frac{1}{2}  h^2  \nabla \log \pi (X) + h Z, 
    \end{equation}
    or equivalently 
    \begin{equation} \label{eq:prop2}
        Y = X + \frac{1}{2}  \epsilon  \nabla \log \pi (X) + \sqrt{\epsilon} Z, 
    \end{equation}
    with $\epsilon = h^2$. In other words, the term in front of the gradient
    should be commensurate to the variance of the noise term.  Our previous code for Figure 5
    implemented this proposal correctly following the parameterization in \eqref{eq:prop2}. 
    In our previous code, \texttt{ula\_stepsize} and \texttt{mala\_stepsize} corresponded to $\epsilon = h^2$ above.
    \item For this example, the optimal scaling for MALA corresponds to $h \sim d^{-1/6}$ or $h^2 \sim d^{-1/3}$.
    However we got it wrong in the implementation: we chose $\epsilon = h^2 \sim d^{-1/6}$, leading to a decaying acceptance rate for MALA as dimension grew, i.e. a sub-optimal tuning of MALA.
\end{itemize}

\section{Updated code and experiments}
We have updated the script
\href{https://github.com/niloyb/LlagCouplings/blob/master/ula_versus_mala/ula_versus_mala.R}{ula\_versus\_mala/ula\_versus\_mala.R}. 
In the updated script,  \texttt{ula\_stepsize} and \texttt{mala\_stepsize} correspond to $h$ in \eqref{eq:prop1}, 
and we choose $h \sim d^{-1/6}$. 

\subsection{Choice of step-size}
We experiment with step-sizes of the form $h=C d^{-1/6}$ for various $C$ and $d$.
We obtain Figure \ref{fig:acceptancerate}, confirming that the MALA acceptance
rate is stable with $d$.

\begin{figure}[!ht]
\begin{center}
\includegraphics[scale=0.6]{mala_acceptrate2.pdf}
\caption{Acceptance rate of MALA for various $C$ and $d$,
  with \texttt{mala\_stepsize} $h=Cd^{-1/6}$. \label{fig:acceptancerate}}
\end{center}
\end{figure}

\subsection{Upper bounds on mixing times}

With the corrected step-sizes, we can again estimate upper bounds on
$t_{\text{mix}}(\delta)$, say with $\delta = 0.25$.  We obtain Figure
\ref{fig:mixing}, based on $50$ independent meetings for each configuration
(dimension, step-size, algorithm). The upper bounds associated with ULA tend to be smaller.
However they provide information on the convergence of ULA to its limiting
distribution, which is not $\pi$. And, they are simply upper bounds, so the
actual mixing times could be ordered differently.

\begin{figure}[!ht]
\begin{center}
\includegraphics[scale=0.6]{ula_mala_mixing2.pdf}
\caption{ Upper bounds on the mixing time, for various dimensions
  and various step-sizes of the form $C d^{-1/6}$,
  for ULA and for MALA. \label{fig:mixing}}
\end{center}
\end{figure}


\bibliographystyle{abbrvnat}
\bibliography{LlagCouplings_biblo.bib}

\end{document}
